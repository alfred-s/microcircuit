\section{Introduction}
\label{sec:intro}
\subsection{Hypotheses}
- The original model can be implemented in Pynest, yielding a much more 
comprehensible structure compared to the SLI implementation. 

- Some of the results of simulating the model can be approximated with a 
mean field theory (most notably, firing rates and membrane potential
distributions).

- The remaining disagreement between simulation and model can be tracked 
down to correlations between spike trains.

- Else: Signal transmission? Autocorrelation???

\subsection{Model derivation}
The full-scale spiking model is based on the derivation and implementation 
by Potjans and Diesmann \cite{2014}, seeking to account for the detailed 
structure of the neocortical microcolumn reported in experimental data. 
Most notably, the feature of various horizontal layers is recaptured, 
separating the simulated nodes into four layers, each containing an 
excitatory and an inhibitory population. 

However, it remains a minimal model in the sense that the number of 
parameters included is small. The neurons are modeled as spiking leaky 
integrate-and-fire neurons, connected by static synapses of one type. 
The connections between single nodes are determined randomly based on 
connection probabilities for each corresponding pre- and postsynaptic 
population and a fixed probabilistic rule. Excitatory and inhibitory 
neurons are distinguished by a global inhibition dominance ratio, following
the approach of balanced random network models.

The model is expected to operate in a stable regime of asynchronous irregular 
(AI) spiking activity \cite{brunel2000}. HOWEVER, THIS NEEDS TO BE INVESTIGATED!

A mean field theory for balanced random network models has been derived by 
Brunel \cite{brunel2000} for one excitatory and one inhibitory population. 
The approach determines analytically some of the most relevant characteristics
of the corresponding simulated networks, such as firing rates, coefficient 
of variation of interspike intervals and membrane potential distributions, 
as well as the power spectrum of global activity. It remains unclear, however, 
how far these results can be applied to more complex networks, containing 
larger number of populations. This study sets out to address these questions. 

\subsection{Structure}
The thesis is structured as follows. First, the implementation of Potjans and 
Diesmann's model in Pynest is presented in detail. The results of simulations 
are compared to the results running the original implementation with identical 
parameters. Then, the mean field theory for two populations is generalized 
to the network model. The parameters predicted with this model are compared 
to the results of the simulation. Finally, the shortcomings of the mean field 
approach are analyzed by questioning the validity of the underlying assumptions
for the case of the network model. 



