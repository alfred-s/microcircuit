\section{Introduction}
\label{sec:intro}

HYPOTHESES

- The original model can be implemented in PyNEST, yielding a 
comprehensible structure.

- Some of the results of Potjans' model can be approximated utilizing a 
mean field theory (firing rates, CV of ISI and membrane potential
distributions).

\emph{- The deviations can be explained partly by the remaining differences 
in parameters, partly by correlations (temporal and between neurons).}

-----------------------------------------------------\\

%Biology -> questions -> models / simulation -> mean field

%Biology
%Neocortex: 
%enable higher cognitive function
%six-layered organization, 
%regionalization into areas 
%(sensory, motor, associative)
%\cite{lui2011development}
%microcircuit
%signal processing: input to layer IV, ...
%rates, oscillations

%Connectivity map
%local connectivity data (different animals, different methods)
%long range neglected

%Spiking network model
%leaky integrate and fire neuron -> good model because?!?
%current based synapses
%sparse random connectivity: network characterized by few parameters
%NEST
%reproduces rates

%Mean field
%analytical framework

%underlying questions: 
    %what drives the network?
    %is it stable?
    %can / are information processed via rates?

%more technical:
    %can measures of data from spiking network simulations be predicted?
    %is this a useful tool for exploring states of the network?

%origin:
%series of papers by Brunel, culminating in a central paper: Brunel2000 
%prework: Tuckwell, Risken 


%It remains unclear, however, 
%how far these results can be applied to more complex networks, containing 
%larger number of populations. This study sets out to address these questions. 


The thesis is structured as follows. The first section contains a detailed account 
of the spiking network model as well as a derivation of the mean field model 
for eight neuron populations.  
The results are then presented starting by comparing the simulation results to 
those of the implementation of the original publication by Potjans and 
Diesmann~\cite{potjans2014} as well as a closer look on the statistical properties 
of spike trains within one population. 
This is followed by comparing simulation results to those
obtained with the mean field model. In order to assess the differences between 
predicted and measured quantities, 
further simulations with parameters adjusted 
to the mean field model are evaluated, 
\emph{showing the strong/weak dependence on the assumptions made for the mean
field theory.}

\emph{Should the hypotheses by subsumed in this last paragraph (i.e. as part of the structure?)
This is in a way what Brunel does (itemized)}


