\chapter{Results}
\label{sec:results}

\section{Spiking network model}
\paragraph{The spiking network model} will be analyzed in the following manner: 
At first, the results are directly compared to those obtained by the original 
model of Potjans and Diesmann. Both simulations are run for the same parameters, 
differences arise solely due to internal differences in assigning the random 
number generators. It is therefor not feasible to do a direct (spike per spike) 
comparison. Instead, a statistical description is chosen which furthermore sheds light
on the differences between individual realizations with different seeds for the 
random number generators.
The distribution of single neuron firing rate within the population
and the interspike interval distributions.
are then analyzed more thoroughly. 

A comparison between the PyNEST implementation and the original one written in SLI
is supplied in form of raster plots (\autoref{fig:raster_plot}), 
showing the activity of a subset of the network simulated for 400\,ms 
after a transient period of 200 ms. The subset shown consists of the first 
$2.5\,\%$ of the neurons of each population created during the initialization of the network. 
Both results have a very similar structure: 
The corresponding subsets show about the same number of events per time. This 
number varies strongly between different populations. Specifically, populations L2/3e and L6e 
fire at much lower frequencies compared to the remaining populations. 
Furthermore, both simulations show some signs of synchrony within populations, indicated by spikes 
of different neurons aligned in vertical lines. This is most apparent for the populations L2/3e, L2/3i, L4e and 
L5e.
\begin{figure}[tb]
    \centering
    \includegraphics{\figdir raster_plot}
    \caption[Raster plot: PyNEST and SLI]{
        Raster plot showing spontaneous activity of network for 
        (A) the PyNEST implementation and (B) the SLI implementation.
        The simulation and network parameters for both simulations are 
        the same. 
        For each layer, the excitatory population is the upper one shown 
        (total of 1924 neurons) for 400\,ms. 
    }
    \label{fig:raster_plot}
\end{figure}

A more thorough comparison is done on the bases of three statistical quantities: 
\autoref{fig:spontaneous_activity} shows the population means of single neurons firing rates, 
irregularity as measured by the CV of ISI as well as a measure for synchrony (the Fano factor of the 
PSTH) for both simulations (see Methods, \autoref{subsec:analysis} for details). 
All parameters were calculated from $20$ repetitions of a simulation for 
$60$ s.
Spikes were recorded from $1000$ neurons of each population, with recording 
starting after a transient period of $0.2$ s. 
For all three quantities, both implementations are in full agreement for the statistical fluctuations
observed. Fluctuations between different initializations differ for each quantities: For the firing rates, 
the largest variance is observed for L5e. For the irregularity, fluctuations are much smaller with the 
highest ones seen in populations L2/3e and L6e. Finally, the variation for synchrony is largest for 
L2/3e and L5e. For synchrony and firing rates, there is a tendency for the fluctuations to be larger for larger
values of the respective quantity -- note however the exception of L5i firing rate, which shows very little
fluctuation.
\begin{figure}[tb]
    \centering
    \includegraphics{\figdir spontaneous_activity}
    \caption[Spontaneous activity: PyNEST and SLI]{
        Measures of spontaneous activity for
        (\textbf{A - C}) the PyNEST implementation and (\textbf{D - E}) the SLI implementation, 
        using the same simulation and network parameters.
        Both implementations are run $20$ times independently (one marker per simulation),
        measuring 1000 spike trains of each population in a simulation for 60 s.
        In case of small 
        fluctuations, individual markers may not be identified due to overlap. 
        \quad (\textbf{A,~D}) Population mean of single neuron firing rates.
        \quad (\textbf{B,~E}) Irregularity of spike trains measured by the 
        population mean of CV of ISI.
        \quad (\textbf{C,~F}) Synchrony of populations quantified 
        by the Fano factor of the PSTH (bin width 3 ms).
    }
    \label{fig:spontaneous_activity}
\end{figure}

In order to get a deeper insight into the dynamics of the simulated network, the firing 
rates and the CV of ISI of single neurons are examined in \autoref{fig:single_neuron_activity}.
The observed fluctuations around the population mean are remarkably large and can 
be tracked back to the specific connection rule applied.
Note, however, that despite being large within a population, the 
fluctuations of the population means for different realizations remains small -- an observation 
important for the application of a mean field model. Both aspects will be reviewed in the 
Discussion, \autoref{sec:discussion}. 

\begin{figure}[tb]
    \centering
    \includegraphics{\figdir single_neuron_activity}
    \caption[Firing rates and CV of ISI of single neurons]{
        Single neuron activity measures:
        (\textbf{A}) 
        firing rates and
        (\textbf{B}) 
        CV of ISI 
        The data is take from one simulation of the PyNEST implementation, 
        corresponding to a single initialization in \autoref{fig:spontaneous_activity}.
        Statistical fluctuations 
        are indicated by the interquartile ranges (IQR) (boxes extend to 
        the first and third quartile). 
        The median is indicated by a black line, the population mean by a star and 
        whiskers extend to 1.5 IQR (outliers indicated by crosses). 
    }
    \label{fig:single_neuron_activity}
\end{figure}

\section{Mean field theory}
\paragraph{In the following section}, the results of the mean field theory are presented.
by showing the predicted firing rates and utilizing these rates 
to predict the previously introduced measure of irregularity
as well as the distribution of membrane potentials. 
These results are directly compared to 
the corresponding quantities recorded from the spiking network 
model simulation and analyzed in the previous section. 
The average input $\mu$ to each neuron is examined by separating 
the recurrent an external contribution. Finally, the possibility of  
applying the mean field model as a predictive tool is illustrated 
by calculating firing rates for varying relative inhibitory synapse 
strength $g$ and external frequency $\nu_\text{ext}$. 

The firing rates obtained by solving 
equation~\eqref{eq:self_consistency_a} are displayed in a bar plot in 
\autoref{fig:compare_sim_mf_fixed_total_number}. Rates measured in 
simulation and previously shown in \autoref{fig:spontaneous_activity}
are shown for comparison. As visible, the results of the mean field model 
match those of the simulation to a high degree:

The predicted rates cover those observed in simulation for the populations 
of all layers by L5. For the latter one, the rate of the excitatory population
is underestimated, the inhibitory one predicted slightly too high. 
Qualitatively, the sequence of populations ordered by increasing firing rates is reproduced.
The comparison can further be quantified: The difference 
$    \Delta \nu_a := \nu_{\text{mf}, a} - \nu_{\text{sim}, a} $
between the mean of simulated rates $\nu_{\text{sim}, a}$ and predicted rates 
$\nu_{\text{mf}, a}$ for each population $a$ is shown in \autoref{tab:diff_fixed_total_number}. 
The mean and standard 
deviation of the absolute values of $\Delta \nu_a$ 
are $(0.22 \pm  0.16)$ Hz. The relative difference
is largest for L2/3e with $-11\,\%$, while for all other populations 
the relative difference is smaller or equal $7\,\%$. 
The irregularity measured by the mean CV of ISI for each population is the result of 
plugging in the predicted rates into equation~\eqref{eq:CV_ISI_mf}. 
As for the rates, a comparison with simulation data is shown
in \autoref{fig:compare_sim_mf_fixed_total_number}, 
while numerical results are subsumed in 
\autoref{tab:diff_fixed_total_number}, using definitions analogous to those for the rates.
The theoretical results agree well: The relative difference between measured and
predicted values is lower than $10\,\%$ for all populations and again the order 
by sorting the populations according to increasing CV of ISI is reproduced. 
There is, however, a systematic overestimation of irregularity for all populations. 
The largest deviation is found for the populations L2/3e and L6e. For both populations, 
the mean field theory predicts a CV of ISI of almost one, corresponding to the case of 
a Poisson process. 

% Table of results of comparison
\begin{table}[htb]
    \centering
    \caption[Differences between prediction and simulation]{
        Difference between predicted and simulated population means for 
        firing rates and CV of ISI; absolute and relative to simulated quantities.}
    \label{tab:diff_fixed_total_number}
    \small
    \begin{tabular}{p{2.4cm} *{8}{x{0.87cm}}} \toprule
    \rowcolor{TableColor}
    \spacedlowsmallcaps{Population}
        & \mc2c{L2/3} & \mc2c{L4} & \mc2c{L5} & \mc2c{L6}  \tn
        \rowcolor{TableColor}
        & \mc1c{e} & \mc1c{i} & \mc1c{e} & \mc1c{i} & \mc1c{e} & \mc1c{i} & \mc1c{e} & \mc1c{i} \tn %\hline
        %Population $a$       
        %& L2/3e & L2/3i & L4e & L4i & L5e & L5i & L6e & L6i  \tn[0.2cm]
        \rule{0.0cm}{3ex} 
        $ \nu_{\text{mf}, a}$ / Hz
            &   0.82 &   3.02 &   4.64 &   6.12 &   7.14 &   8.92 &   1.04 &   8.09 \tn[0.2cm]
        $ \nu_{\text{sim}, a}$ / Hz
            &   0.92 &   3.00 &   4.40 &   5.84 &   7.70 &   8.65 &   1.10 &   7.84 \tn[0.2cm]
        $\Delta \nu_a$ / Hz
            &  -0.10 &   0.01 &   0.24 &   0.28 &  -0.56 &   0.27 &  -0.06 &   0.24 \tn[0.2cm]
        $\Delta \nu_a / \nu_{\text{sim}, a}$
            &  -0.11 & 4e-3&   0.05 &   0.05 &  -0.07 &   0.03 &  -0.05 &   0.03 \tn[0.4cm]
        $ \text{CV}_{\text{mf}, a}$
            &   0.99 &   0.94 &   0.92 &   0.91 &   0.89 &   0.84 &   0.99 &   0.85 \tn[0.2cm]
        $ \text{CV}_{\text{sim}, a}$
            &   0.92 &   0.92 &   0.89 &   0.88 &   0.84 &   0.81 &   0.91 &   0.81 \tn[0.2cm]
        $\Delta \text{CV}_a$
            &   0.07 &   0.03 &   0.03 &   0.03 &   0.05 &   0.04 &   0.08 &   0.04 \tn[0.2cm]
        $\Delta \text{CV}_a / \text{CV}_{\text{sim}, a}$
            &   0.07 &   0.03 &   0.03 &   0.04 &   0.06 &   0.05 &   0.08 &   0.05 \tn[0.2cm]

        \bottomrule
    \end{tabular}
\end{table}

% Comparison mean field / simulation
\begin{figure}[tb]
    \centering
    \includegraphics{\figdir compare_sim_mf_fixed_total_number}
    \caption[Comparing mean field model to simulation]{
        Comparison between mean field theory and spiking network model. 
        Bars indicate 
        (\textbf{A}) the single neuron firing rates and
        (\textbf{B}) the irregularity (mean CV of ISI)
        predicted by the mean field 
        theory, crosses the respective measurements from 20 simulation (as previously shown in
        \autoref{fig:spontaneous_activity}). The connection for the simulations
        rule was set to "fixed total number".
    }
    \label{fig:compare_sim_mf_fixed_total_number}
\end{figure}

Applying equation~\eqref{eq:P_V_a} for the predicted rates yields the 
membrane potential distributions shown in 
\autoref{fig:membrane_potential}. 
The obtained distributions are compared with the normalized histograms of recorded 
membrane potentials of a subpopulation of $n_\text{rec} = 100$ neurons for 
each population. The contribution due to neurons in refractory period is removed
(see Methods, \autoref{subsec:analysis} for details). 
The predictions agree well with the measured data: The overall shape, width and 
height of the distributions are recovered. For all populations, the maxima are
shifted toward the resting potential $V_\text{r} = -65$ mV. 
The effect
of neurons coming out of refractory period is underestimated in some cases, 
visible for example in populations L2/3i and L4i where the mean distributions 
show a step while the kink in the theoretical curves is hardly detectable. 

% Membrane potentials
\begin{figure}[tb]
    \centering
    \includegraphics{\figdir membrane_potential}
    \caption[Distribution of membrane potentials]{
        Distribution of membrane potentials for each population. 
        Shown are both the results of simulation (histogram) and 
        the predictions of the mean field theory (continuous line). 
        The simulation results are histograms (bins width $\Delta V_\text{m} = 0.25$ mV) 
        of membrane potential recordings 
        of 100 neurons, recorded every 0.001\,s for a simulation time of 1.0\,s 
        and adjusted for neurons in refractory period (see text). 
        The binning in voltage is the same as applied in Fig.~%
        \autoref{fig:single_membrane_potential}. 
        The voltage for neurons in refractory period $V_\text{r} = -65$\,mV 
        is indicated by the dashed and dotted line. The threshold is at 
        $\theta = -50$\,mV. 
    }
    \label{fig:membrane_potential}
\end{figure}

\paragraph{To illustrate} the average input $\mu$ to a neuron, the summands of the local input, cf.~%
equation~\eqref{eq:mu_a_plus}, the total local and external input $\mu_\text{local}$ and 
$\mu_\text{ext}$ as well as their sum are shown in \autoref{fig:input}.
One observes that the recurrent input is concentrated
mostly within layers, either on the diagonal or neighboring it. 
A notable exception is the input of layer 4 to neurons of layer 2/3.
Summing each row yields the total local input $\mu_\text{local}$ per neuron 
for each population. It is inhibitory for each population. L6e is inhibited most strongly, 
which is in agreement with the low firing rate. However, this observation has only
    limited meaning, as both external rates as well as fluctuations are not taken 
into account. The external input $\mu_\text{ext}$ balances the recurrent one, 
yielding positive values for all populations. In agreement with the firing rates, 
population L2/3e and L6e receive very little mean input. Contrary to this, 
population L6i receives even slightly more than L5i ($7.9$\,mV as opposed to $7.8$\,mV), 
while the latter one fires with a higher frequency. This shows the importance
of including the fluctuations. 

\begin{figure}[tb]
    \centering
    \includegraphics{\figdir input}
    \caption[Mean input $\mu$]{
        Mean input $\mu$ to neurons and separated 
        constituents (see equation~\eqref{eq:mu_a_plus}).
        (\textbf{A}) The local input of each presynaptic population $a$ 
        (columns) to a neurons op population $b$ (rows, postsynaptic). 
        The first column in (\textbf{B})
        corresponds to the total recurrent input $\mu_\text{local}$
        and equals the sum of each row in (\text{A}). The second column is 
        the external input $\mu_\text{ext}$.
        (\textbf{C}) Total input $\mu = \mu_\text{local} + \mu_\text{ext}$.
        Note the different scales between (\textbf{B}) and (\textbf{C}). 
    }
    \label{fig:input}
\end{figure}

We can now vary specific parameters in the mean field theory
and predict the network activity for the resulting new 
model. This is interesting, as a model used for explaining 
experimental data should be expected to be rather robust against
changes in specific parameters. One parameter that is estimated 
very differently in different theoretical models 
is the inhibitory synaptic 
strength $g$ (\citeb{sadeh2014mean} for example uses $g = 8$). 
In \autoref{fig:simulate_change_g}, the firing rates are calculated for $g$ 
varying over the range from $3$ to $10$. A first observation is the general
decrease of firing rates with increasing $g$. The individual populations, 
on the other hand, do not decrease at the same rate, such that the characteristic order 
of firing rates is changed. This is especially the case for population L4e, 
which stays almost constant on the entire range (remaining at the highest rate 
for very strong inhibition dominance) and L2/3e, for which the rate even 
increases. The strongest decrease is observed for layer L5e.
Note that \citeb{potjans2014} apply this change in $g$ in simulation. 
The data shown (Figure 8 of \cite{potjans2014}) agrees with the predictions
of the model, although comparison is limited to excitatory populations. 
For $g$ lower than $3$, the applied algorithm does not find a solution 
any more. For very low inhibition, however, the mean field model's assumptions
are not necessarily met any more, since individual firing rates tend to rise. 

\begin{figure}[tb]
    \centering
    \includegraphics{\figdir simulate_change_g}
    \caption[Firing rates for different $g$]{
        Firing rates for different inhibitory synaptic strength $g$
        predicted by the mean field theory. The working point of 
        the spiking network simulation is at $g = 4$, other theoretical
        models use higher estimates up to $g = 8$. 
    }
    \label{fig:simulate_change_g}
\end{figure}


\FloatBarrier
