\section{Results}
\label{sec:results}

\subsection{Spiking network model}
The simulation of the spiking network model will be analyzed in the following manner. 
At first, the results are directly compared to those obtained by the original 
model of Potjans and Diesmann. This comparison is done on a statistical level, 
as direct (spike per spike) comparison is not feasible due to internal differences
in the algorithms. 
Then, some aspects of the results are analyzed more thoroughly, such as 
the interspike interval distributions, global activity and 
membrane potential distribution. These results are compared with 
theoretical approximations such as the distributions expected for Poisson processes. 

A comparison between the pyNEST implementation and the original one written in SLI
is first supplied in form of raster plots, see Figure \ref{fig:raster_plot}. 
Both results have a very similar structure: Common features are the observable low 
rates of populations $L2/3e$ and $L6e$ as well some signs of synchrony within populations. Single neurons fire at a rate much lower than the activity of the 
entire population, as would be expected for networks in the AI regime \cite{brunel2000}. 

A more thorough comparison is done on the bases of three statistical quantities: 
Figure \ref{fig:spon_activity} shows single neurons firing rates, coefficient of 
variation of interspike intervals as well as a measure for synchrony for both 
simulations. The latter one is calculated by dividing the variance of population
activity subdivided in bins of 3 ms by its mean. 


\emph{SHORT EXPLANATION OF SHOWN QUANTITIES:\\
Raster plot, box plot of single neuron firing rates, CV of ISI, synchrony.\\
FURTHER: \\
Plots directly comparing these quantities;\\
Plots for statistical fluctuations -> box plots} 
\begin{figure}[htpb]
    \centering
    \includegraphics[width=1.0\linewidth]{\figdir raster_plot}
    \caption{Raster plot showing spontaneous activity of network for 
        (A) the pyNEST implementation and (B) the SLI implementation.
        The simulation and network parameters for both simulations are 
        the same. 
        For each layer, the excitatory population is the upper one shown 
        (total of 1924 neurons) for $400$ ms (with a transient period of 100 ms). 
    }
    \label{fig:raster_plot}
\end{figure}
\begin{figure}[htpb]
    \centering
    \includegraphics[width=1.0\linewidth]{\figdir spontaneous_activity}
    \caption{
        Derived statistical quantities for spontaneous activity of network for
        (\textbf{A - C}) the pyNEST implementation and (\textbf{D - E}) the SLI implementation, 
        using the same simulation and network parameters.
        All quantities are derived by measuring 1000 spike trains of
        each population in a simulation for 60 s. 
        \quad (\textbf{A, D}) Boxplot of single neuron firing rates. The boxes show the extends of 
        Q1 and Q3, median is indicated by a black line, whiskers extend to 1.5 IQR. 
        Outliers are displayed as plus-signs. 
        \quad (\textbf{B, E}) Coefficient of variation (CV) of interspike intervals (ISI) indicating 
        the irregularity of single neuron spiking. 
        \quad (\textbf{C, F}) Synchrony of the recorded subset of each population quantified by the 
        variance of the summed spike count histograms (bin width 3 ms) divided by
        its mean. 
    }
    \label{fig:spontaneous_activity}
\end{figure}

More detailed view on results:
Synchrony of populations in terms of population activity, Figure \ref{fig:population_activity}. 
The observed global oscillation indicates non-negligible correlations
and thus contradicts the assumption of uncorrelated gaussian white noise 
(in the mean field model).
\begin{figure}[htpb]
    \centering
    \includegraphics[width=0.8\linewidth]{\figdir population_activity}
    \caption{Population activity of all eight populations.
        (A) Number of spikes per 1 ms over the time of 1 s;
        (B) Power spectral density (PSC) of population activity minus mean activity. 
        Lighter and heavily oscillating background is the complete PSD, thick darker 
        line represents filtered data (Savitzky-Golay filter of 5th degree over 301 data points).
        The peaks indicate global oscillations at $\approx 82$ Hz (origin not clear).
    }
    \label{fig:population_activity}
\end{figure}



\subsection{Mean field theory}
The results regarding the mean field approach consist of a solution to the problem 
of convergence of the numerical solving and a direct comparison of the predicted rates
with the ones obtained in the spiking network model simulation.

Transition: \\
Rates for Brunel's model known/ much easier to solve (only one rate) \\
Step-wise transition from Brunel's model to Potjans, see Figure \ref{fig:transition_mf_vs_model} \\
Finally: \\ 
differences between modeling with "fixed indegree" and "fixed total number",
figures \ref{fig:compare_sim_mf_fixed_indegree} and 
\ref{fig:compare_sim_mf_fixed_total_number}.
\begin{figure}[htpb]
    \centering
    \includegraphics[width=0.8\linewidth]{\figdir transition_mf_vs_model}
    \caption{Transition from Brunels model of equal populations to the neocortical network model. 
        Single neuron firing rates obtained by simulation (dotted lines) and 
        numerical solution of the mean field model. Distance $d$ indicates the 
        normalized $L^2$-norm in the parameter space ($8 \times 8$ synapse numbers $+ j_{02}$, see text
        for further explanation). 
    }
    \label{fig:transition_mf_vs_model}
\end{figure}
\begin{figure}[htpb]
    \centering
    \includegraphics[width=0.8\linewidth]{\figdir compare_sim_mf_fixed_indegree}
    \caption{Comparison between mean field theory and spiking network model. 
        Bars indicate the single neuron firing rates predicted by the mean field 
        theory, crosses the calculated ones for 20 simulations where the connection
        rule was set to "fixed indegree".
    }
    \label{fig:compare_sim_mf_fixed_indegree}
\end{figure}
\begin{figure}[htpb]
    \centering
    \includegraphics[width=0.8\linewidth]{\figdir compare_sim_mf_fixed_total_number}
    \caption{Comparison between mean field theory and spiking network model.
        Similar to Figure \ref{fig:compare_sim_mf_fixed_indegree} except for the connection 
        rule which was set to "fixed\_total\_number". The measured rates of the simulation 
        fluctuate more strongly but around the same mean. 
    }
    \label{fig:compare_sim_mf_fixed_total_number}
\end{figure}
