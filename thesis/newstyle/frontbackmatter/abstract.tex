%*******************************************************
% Abstract
%*******************************************************
%\renewcommand{\abstractname}{Abstract}
\pdfbookmark[1]{Abstract}{Abstract}
\begingroup
\let\clearpage\relax
\let\cleardoublepage\relax
\let\cleardoublepage\relax

\chapter*{Abstract}
In the search for understanding the basic functions of the neocortex, 
linking the anatomical structure to population dynamics has been a mayor
focus of research. Experimental data in form of local connectivity data
and cell-type specific activity is increasing at a fast past but remains
mainly inconclusive. Simultaneously, spiking network models using 
leaky integrate-and-fire neurons 
have been successfully developed and used for interpreting 
statistical features in experimental data such as firing rates, 
correlations or oscillations. 
On the one hand, these models can be implemented in numerical 
simulations such that specific hypotheses can be tested. 
On the other hand, a deeper understanding is reached by 
an analytical framework relying on a statistical description.
Such a rate based mean field theory has been developed 
for networks of two populations
by Brunel~\cite{brunel2000} and successfully applied for a number 
spiking network models. However, for more complicated networks
it is not a priori clear whether this description is adequate. 
One important case is the modeling of the local cortical 
microcircuit, featuring a laminar structure. 
A full-scale spiking network model incorporating this characteristic
was established by Potjans and Diesmann~\cite{potjans2014} in 2014, 
integrating a large number of the experimental studies available and reproducing
some of the main features of spiking activity of the cortex.
In this work, the spiking network model of the microcircuit is reimplemented, 
compared with the original one and further analyzed.
Aiming for a more thorough understanding as well as a computationally less
expensive tool, the existing mean field theory is extended to the given network. 
The predictions, namely single neuron firing rates, 
membrane potential distributions as well as the irregularity of spike trains, 
are tested against simulated data. 
%\emph{Finally, in order to test 
%the dependence of a number of parameters, a simulation adapted to the 
%mean field theory is contrasted with the previous results.}

\vfill

\pdfbookmark[1]{Zusammenfassung}{Zusammenfassung}
\pagebreak
\chapter*{Zusammenfassung}
Kurze Zusammenfassung des Inhaltes in deutscher Sprache\dots


\endgroup			

\vfill
