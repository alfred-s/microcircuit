\chapter{Discussion}
\label{sec:discussion}

% Spiking network
\paragraph{The reimplementation} of Potjans and Diesmann's spiking network simulation  
confirmed the expectation: The original model's results were reproduced 
well within the statistical fluctuations. The fluctuations of the calculated population 
means between different instantiations of the model are 
relatively small, a characteristic that can be interpreted as an indication that a mean 
field approach is suited well for describing the corresponding network:
The parameters describing the main features of the network activity
depend strongly on average input and connection numbers,
and much less on the actual wiring.
However, this assertion has to be restricted to 
random networks, as more detailed structure, for example induce by learning, 
can lead to stronger effects of correlation (see for example \citeb{staude2010higher}).
For a single simulation, the fluctuations along neurons within one population
were shown to be large both for the firing rate and the 
CV of ISI. This can be tracked back to the large deviations of input between 
different neurons: Due to randomly choosing the synapses and adding further 
variability by distributing the synapse number (as shown for model validation, 
see Methods, \autoref{subsec:methods_simulation}), 
some neurons will receive more excitatory or inhibitory input than others 
and fire accordingly. 
Since the results reproduce those of the original study, the according characterization 
of the network holds as well: The network activity is labeled as asynchronous irregular, 
with a mean CV of ISI above 0.8 and the applied synchrony measure ranging from 
values below $2$ (populations L5i, L6e and L6i), up to $\sim 8$ for L5e.
Still, the larger values of synchrony indicate that the amount of correlation
might not be negligible for the spike rate. This has to be taken 
into account when interpreting the results of the mean field model. 

% Mean field theory
\paragraph{Introducing the mean field theory} for the simulated spiking network 
turned out to be remarkably successful. 
The formal extension 
from the original model of two populations has been a rather small step, 
while the incorporation of further details as well as the numerical implementation 
revealed a number of obstacles. The resulting
algorithm, nonetheless, is a convenient and computationally inexpensive tool for 
predictions. It has been shown to predict central quantities 
of the network activity to a high degree of accuracy.  

% MF: Rates
The predicted single neuron firing rates differ by less than
$0.3$\,Hz for all populations but L5e ($0.6$\,Hz lower than measured). 
Except for the latter and L2/3e, the relative error is below $7\,\%$.
In most cases, the rates are slightly lower than the measured ones..
One reason for this underestimation could be the negligence of correlation: 
Larger correlations among excitatory input can lead to higher fluctuations 
and thus higher spiking rates~\cite{staude2010higher}. This 
assertion has to be taken with care, though, as correlation in inhibitory 
populations may cancel the effect as shown e.\,g. by \citeb{renart2010asynchronous}. 

% MF: CV of ISI
The prognosis for the irregularity of spike trains, measured by the coefficient of
variation of interspike intervals (CV of ISI) turns out to be evenly accurate:
For all populations, the deviation between theory and simulation is of the order 
of $0.05$, the largest being observed for the populations L2/3e and L6e with $0.07$ 
and $0.08$. The latter ones are the populations firing at the lowest rate. 
The general overestimation of irregularity can be interpreted in two ways. 
One explanation for the remaining differences can be an estimation bias of the CV of ISI 
using spike trains of finite lengths arising as a significant ratio of the neurons in the 
population is either not included at all (if the number of spikes is $ < 2$) or the part of the 
distribution covering higher CV of ISI is not represented well (see \citeb{nawrot2010analysis} for 
details). This is especially critical for the populations with low rates (L2/3e, L6e), 
which indeed do show the largest deviations.
What cannot be accounted for by this bias would indicate a lower irregularity than that 
of the respective stationary Gaussian process. This points towards temporal correlations
introduced for example by the synapse model beyond the effects accounted for
(see e.\,g. \citeb{brunel1999fast} for the case of $\alpha$ synapses in a simpler context). 

% MF: Membrane potentials
The third and last measure predicted by the mean field model, the distribution of 
membrane potentials, also agrees well with the obtained simulation data. 
Both the shape and position of the distribution are reproduced.
The kink at the resting potential $V_\text{r}$ due to neurons exiting the 
refractory period is reproduced but less pronounced than the measured one. 
This indicates that the diffusion
away from this point is slower than assumed. 
The maxima of the predicted distributions are slightly 
shifted towards the resting potential. 
Again, the interpretation of the distribution has to be done with care:
A lower maximum does not correspond to lower firing rates. 
The populations of layer 5 illustrate this: While for both, the shift in membrane 
potential is about equal, the excitatory firing rate is underestimated whereas the 
inhibitory one is overestimated. 
In the mean field model, this is reflected by relating the firing rate 
only to the probability current at the threshold $\theta$, not to the 
shape of the curve at other points (cf. \autoref{eq:prob_curr} and following).
Further, higher firing rates due to correlations are not detectable in the 
membrane potential distribution: 
Even if the membrane potential spends most time close to the resting potential, 
a large number of excitatory spikes arriving in a short time would lead to a quick rise and
firing without affecting the distribution significantly.  

% MF: test
The example for applying the mean field range for a range of different inhibitory 
synapse strengths $g$ showed that it is a convenient tool for predicting 
network behavior. This is especially of use since simulating a network is tied 
to high computational costs and waiting time. When applying the model in this manner, 
the underlying assumptions have to be kept in mind in order to not leave the 
range of validity. 

% Summary of results, explanations and shortcomings
\paragraph{In summary,} the hypothesis for the analytical ansatz is confirmed: 
The considered activity measures of the simulated spiking network model of the neocortex 
can be predicted by a mean field theory assuming uncorrelated Gaussian input.
Possible explanations for the remaining deviations are (i) correlations between neurons, 
yielding an input to single neurons different from the assumed white Gaussian noise;
(ii) temporal correlations induced by the synapse type different from the 
delta synapses of the mean field model; and finally (iii) fluctuations in the 
input of single neurons. 
%\emph{Comparison with simulations where the according parameters are adapted 
    %have show the individual effects}.

% Use of the theory, limits
The value of the analytical framework developed is twofold: On the one hand, 
it provides a useful tool for predicting the activity of a spiking network model
over a large range of parameters 
%\emph{as observed for external frequency $\nu_{ext}$ 
%and relative inhibitory synapse strength $g$}. 
On the other hand, it represents an
essential means for identifying relevant measures and understanding the emergent 
dynamics of the complex systems under consideration. This is a hard and one of the 
most important tasks in this field. 
The long lasting
debate over whether neural coding is rate based or exploits precise timing and correlations
may at some point be solved by
excluding one or the other option using a sensible framework
of biological data, spiking network simulation and analytical arguments. 
On a less ambitious scale, the presented models can at least indicate by how much 
correlations effect the observed rates. 

% Outlook
The presented combination of the spiking network model and the mean field approach
serves as a framework to tackle further questions. 
The implementation in PyNEST works as a convenient basis for extensions 
such as the inclusion of newly available experimental data.
Furthermore, different neuron populations, 
especially concerning different interneuron classes, may be included, 
making the simulation a viable means for testing hypothesis about their 
role. 
Another possible extension already introduced in the 
original model \cite{potjans2014} is unspecific input from a thalamic population. 
When applying this input for short bursts (e.\,g. 10 ms), 
this can by used in order to examine the way
increased spike rates are propagated 
along the different layers in time and thus assessing the 
path of information processing within the neocortex.
The mean field approach, however, would have to be extended to a non-equilibrium 
regime since it lacks temporal resolution at the present state. 
Finally, when focusing on neural computation, one might also include specific input.
An interesting context is orientation selectivity, using 
oriented input resulting in neuronal tuning curves.
To this end, the mean field approach can be extended to single neurons 
as implemented for example by ~\citeb{sadeh2015orientation} 
for the case of one excitatory and one inhibitory population.




