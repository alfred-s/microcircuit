%%%%%%%%%%%%%%%%%%%%%%%%%%%%%%%%%%%%%%%%%
% Short Sectioned Assignment
% LaTeX Template
% Version 1.0 (5/5/12)
%
% This template has been downloaded from:
% http://www.LaTeXTemplates.com
%
% Original author:
% Frits Wenneker (http://www.howtotex.com)
%
% License:
% CC BY-NC-SA 3.0 (http://creativecommons.org/licenses/by-nc-sa/3.0/)
%
%%%%%%%%%%%%%%%%%%%%%%%%%%%%%%%%%%%%%%%%

%----------------------------------------------------------------------------------------
%	PACKAGES AND OTHER DOCUMENT CONFIGURATIONS
%----------------------------------------------------------------------------------------

\documentclass[paper=a4, fontsize=12pt, xcolor=dvipsnames, parskip=full]{scrartcl} % A4 paper and 11pt font size

% Biblatex
\usepackage[
style=nature,              % Zitierstil
isbn=false,                % ISBN nicht anzeigen, gleiches geht mit nahezu allen anderen Feldern
pagetracker=true,          % ebd. bei wiederholten Angaben (false=ausgeschaltet, page=Seite, spread=Doppelseite, true=automatisch)
maxbibnames=50,            % maximale Namen, die im Literaturverzeichnis angezeigt werden (ich wollte alle)
maxcitenames=3,            % maximale Namen, die im Text angezeigt werden, ab 4 wird u.a. nach den ersten Autor angezeigt
autocite=inline,           % regelt Aussehen für \autocite (inline=\parancite)
block=space,               % kleiner horizontaler Platz zwischen den Feldern
backref=false,              % Seiten anzeigen, auf denen die Referenz vorkommt
backrefstyle=three+,       % fasst Seiten zusammen, z.B. S. 2f, 6ff, 7-10
date=short,                % Datumsformat
hyperref=true,
backend=biber
]{biblatex}
\setlength{\bibitemsep}{1em}     % Abstand zwischen den Literaturangaben
\setlength{\bibhang}{2em}        % Einzug nach jeweils erster Zeile
\bibliography{ba_thesis}  % Bibtex-Datei wird schon in der Preambel eingebunden
\usepackage{cmbright}
\usepackage[T1]{fontenc} % Use 8-bit encoding that has 256 glyphs

\usepackage{amsmath, amsfonts, amsthm} % Math packages
\usepackage{pgfplots}
\usepackage{wrapfig}
\usepackage{color, colortbl}  

% Bibliographie auf deutsch
\definecolor{gray1}{gray}{0.2}
\usepackage[font={color=gray1}, figurename=Fig., labelfont={color=blue}]{caption}

\usepackage{titlesec}

\usepackage[utf8]{inputenc} 
\usepackage{forloop}

\usepackage{latexsym}
\usepackage{textcomp}
\usepackage{bm}% bold math
\usepackage{graphicx}
\usepackage{eso-pic}
\usepackage{caption}
\usepackage{subcaption}
\usepackage{verbatim}
\usepackage{epsfig}
\usepackage{framed,color}
\usepackage{placeins}   % FloatBarrier
\usepackage{float}
\usepackage{sidecap}
%\usepackage{float}      % adding boxes around figures
%\floatstyle{boxed}
%\restylefloat{figure}
\usepackage[usenames,dvipsnames]{pstricks}
\usepackage{epsfig}
\usepackage{tikz}
\usepackage{sectsty} % Allows customizing section commands
\usepackage{hyperref}
\usepackage{array}
\usepackage{diagbox, pict2e}


%\usepackage[pass,showframe]{geometry} % just to show the margins
%\usepackage{booktabs}
%\usepackage{minted} % needs '-shell-escape' to compile

\listfiles
\hypersetup{
     colorlinks   = true,
     citecolor    = gray,
     linkcolor    = blue
}

\allsectionsfont{ \color{gray1} \normalfont\scshape} % Make all sections centered, the default font and small caps

\usepackage{fancyhdr} % Custom headers and footers
\pagestyle{fancy} % Makes all pages in the document conform to the custom headers and footers

\renewcommand{\headrulewidth}{0.0pt} % Remove header underlines
\renewcommand{\footrulewidth}{0pt} % Remove footer underlines

\numberwithin{equation}{section} % Number equations within sections (i.e. 1.1, 1.2, 2.1, 2.2 instead of 1, 2, 3, 4)
\numberwithin{figure}{section} % Number figures within sections (i.e. 1.1, 1.2, 2.1, 2.2 instead of 1, 2, 3, 4)
\numberwithin{table}{section} % Number tables within sections (i.e. 1.1, 1.2, 2.1, 2.2 instead of 1, 2, 3, 4)

\setlength\parindent{0pt} % Removes all indentation from paragraphs - comment this line for an assignment with lots of text
\setcapindent{1cm} 

%----------------------------------------------------------------------------------------
%	TITLE SECTION
%----------------------------------------------------------------------------------------
\title{
\normalfont \normalsize 
\textsc{Albert-Ludwigs-University Freiburg} \\ [25pt] % Your university, school and/or department name(s)
\horrule{0.5pt} \\[0.4cm] % Thin top horizontal rule
\huge \textsc{A network model of the neocortex} \\ % The assignment title
\horrule{2pt} \\[0.5cm] % Thick bottom horizontal rule
}

\author{Friedrich Schüßler} % Your name

\date{\normalsize\today} % Today's date or a custom date
\DeclareGraphicsExtensions{.png,.pdf,.jpg,.eps}

%--------------------------------------------------------------------------------------------
% New Commands
%--------------------------------------------------------------------------------------------
\newcommand{\horrule}[1]{\color{gray1}\rule{\linewidth}{#1}} % Create horizontal rule command with 1 argument of height

% Right aligned table cells with fixed length; e.g. x{2.5cm}
\newcolumntype{x}[1]{%
>{\raggedleft\hspace{0pt}}p{#1}}%
\newcommand{\tn}{\tabularnewline}
\newcommand{\tnn}{\tabularnewline[0.3cm]} % newline for costume table cells 

\definecolor{TableColor}{HTML}{A1D99B}

\newcommand{\figdir}{../../analysis/figures/} % figure directory



%----------------------------------------------------------------------------------------
\begin{document}

\color{gray1}
\maketitle
%\begin{center}
% \includegraphics[width=0.6\linewidth]{figures/unifreiburg}
%\end{center}
\thispagestyle{empty}
\newpage
    {\pagestyle{plain}
    \thispagestyle{empty}
    \tableofcontents
    \thispagestyle{empty}
    \cleardoublepage}
\newpage


%----------------------------------------------------------------------------------------
%	CONTENT
%----------------------------------------------------------------------------------------
\setcounter{page}{1}
%*******************************************************
% Abstract
%*******************************************************
%\renewcommand{\abstractname}{Abstract}
\pdfbookmark[1]{Abstract}{Abstract}
\begingroup
\let\clearpage\relax
\let\cleardoublepage\relax
\let\cleardoublepage\relax

\chapter*{Abstract}
In the search for understanding the basic functions of the neocortex, 
linking the anatomical structure to population dynamics has been a mayor
focus of research. Experimental data in form of local connectivity data
and cell-type specific activity is increasing at a fast past but remains
mainly inconclusive. Simultaneously, spiking network models using 
leaky integrate-and-fire neurons 
have been successfully developed and used for interpreting 
statistical features in experimental data such as firing rates, 
correlations or oscillations. 
On the one hand, these models can be implemented in numerical 
simulations such that specific hypotheses can be tested. 
On the other hand, a deeper understanding is reached by 
an analytical framework relying on a statistical description.
Such a rate based mean field theory has been developed 
for networks of two populations
by Brunel~\cite{brunel2000} and successfully applied for a number 
spiking network models. However, for more complicated networks
it is not a priori clear whether this description is adequate. 
One important case is the modeling of the local cortical 
microcircuit, featuring a laminar structure. 
A full-scale spiking network model incorporating this characteristic
was established by Potjans and Diesmann~\cite{potjans2014} in 2014, 
integrating a large number of the experimental studies available and reproducing
some of the main features of spiking activity of the cortex.
In this work, the spiking network model of the microcircuit is reimplemented, 
compared with the original one and further analyzed.
Aiming for a more thorough understanding as well as a computationally less
expensive tool, the existing mean field theory is extended to the given network. 
The predictions, namely single neuron firing rates, 
membrane potential distributions as well as the irregularity of spike trains, 
are tested against simulated data. 
%\emph{Finally, in order to test 
%the dependence of a number of parameters, a simulation adapted to the 
%mean field theory is contrasted with the previous results.}

\vfill

\pdfbookmark[1]{Zusammenfassung}{Zusammenfassung}
\pagebreak
\chapter*{Zusammenfassung}
Kurze Zusammenfassung des Inhaltes in deutscher Sprache\dots


\endgroup			

\vfill

\section{Introduction}
\label{sec:intro}

HYPOTHESES

- The original model can be implemented in PyNEST, yielding a 
comprehensible structure.

- Some of the results of Potjans' model can be approximated utilizing a 
mean field theory (firing rates, CV of ISI and membrane potential
distributions).

\emph{- The deviations can be explained partly by the remaining differences 
in parameters, partly by correlations (temporal and between neurons).}

-----------------------------------------------------\\

%Biology -> questions -> models / simulation -> mean field

%Biology
%Neocortex: 
%enable higher cognitive function
%six-layered organization, 
%regionalization into areas 
%(sensory, motor, associative)
%\cite{lui2011development}
%microcircuit
%signal processing: input to layer IV, ...
%rates, oscillations

%Connectivity map
%local connectivity data (different animals, different methods)
%long range neglected

%Spiking network model
%leaky integrate and fire neuron -> good model because?!?
%current based synapses
%sparse random connectivity: network characterized by few parameters
%NEST
%reproduces rates

%Mean field
%analytical framework

%underlying questions: 
    %what drives the network?
    %is it stable?
    %can / are information processed via rates?

%more technical:
    %can measures of data from spiking network simulations be predicted?
    %is this a useful tool for exploring states of the network?

%origin:
%series of papers by Brunel, culminating in a central paper: Brunel2000 
%prework: Tuckwell, Risken 


%It remains unclear, however, 
%how far these results can be applied to more complex networks, containing 
%larger number of populations. This study sets out to address these questions. 


The thesis is structured as follows. The first section contains a detailed account 
of the spiking network model as well as a derivation of the mean field model 
for eight neuron populations.  
The results are then presented starting by comparing the simulation results to 
those of the implementation of the original publication by Potjans and 
Diesmann~\cite{potjans2014} as well as a closer look on the statistical properties 
of spike trains within one population. 
This is followed by comparing simulation results to those
obtained with the mean field model. In order to assess the differences between 
predicted and measured quantities, 
further simulations with parameters adjusted 
to the mean field model are evaluated, 
\emph{showing the strong/weak dependence on the assumptions made for the mean
field theory.}

\emph{Should the hypotheses by subsumed in this last paragraph (i.e. as part of the structure?)
This is in a way what Brunel does (itemized)}



\section{Methods}
\label{sec:methods}

\subsection{Spiking network simulation}
\label{sub:methods_simulation}
Following Nordlie et al.'s suggestions on 
\emph{good model description practice} \cite{nordlie2009},
the network model is described in prose in the following paragraphs while a detailed 
overview is provided in two tables: The model is summarized in table 
\ref{tab:model_description}, 
whereas specific numerical values of the parameters are displayed in table 
\ref{tab:network_params}. 

\subsubsection{Neuron model}
The neurons are leaky integrate-and-fire neurons with a fixed voltage threshold. 
Below the threshold $\theta$, the dynamics of the membrane potential $V_i(t)$ 
for neuron $i$ are governed by the differential equation 
\begin{equation}
    \tau_\text{m} \,\frac{\text{d} V_i(t)}{\text{d} t} 
            = -(V_i(t) - E_\text{L}) + \frac{\tau_\text{m}}{C_\text{m}} I_i(t) \, .
    \label{eq:leaky_integrator}
\end{equation}
The membrane is specified by its resting potential $E_\text{L}$, 
the time constant $\tau_\text{m}$ and the capacitance $C_\text{m}$.
If at time $t$ the threshold is reached, the neuron emits a spike and remains 
in a refractory period for a fixed time $\tau_\text{rp}$, with the membrane 
potential set to $V_\text{r}$. The total input to the neuron is represented by 
the current $I_i(t)$. 

\subsubsection{Network connectivity}
The network consists of eight cortical populations arranged in four 
layers. Each layer contains an excitatory as well as an inhibitory population. 
A total of 77169 leaky integrate-and-fire neurons are distributed according to the population
sizes given in table \ref{tab:network_params}. The total numbers of excitatory and inhibitory 
neurons are 61843 and 15326, respectively, yielding a ratio of 4.04 of excitatory over inhibitory
neurons. \\
For each combination of pre- and postsynaptic population, pairs of neurons to be connected are drawn
randomly until a fixed number of synapses is reached, allowing for multiple synapses and 
self-connections. Potjans' original model defines the connection probability $P_{\text{conn}, \,ab}$ 
of one neuron in the presynaptic population $a$ to form at least one connection with one neuron in 
the postsynaptic population $b$. For given population sizes $N_a$ and $N_b$, the number of 
synapses $C_{ab}$ is then calculated by
\begin{equation}
    C_{ab} = \frac{\log \left( 1 - P_{\text{conn}, \,ab} \right)}{\log \left( 1 - \frac{1}{N_a N_b} \right)} \, ,
    \label{eq:synapse_numbers}
\end{equation}
the inverse of the formula for connection probabilities derived by Potjans et. al \cite{potjans2014}.

% Model description
\begin{table}[htpb]
    \centering
    \caption{
        Model description according to Nordlie et al. \cite{nordlie2009}. 
        Specific parameters are shown in table \ref{tab:network_params}.
        }
    \label{tab:model_description}
    \begin{tabular}{m{3.1cm} p{10cm}}
        \rowcolor{TableColor}\multicolumn{2}{l}{Model summary} \\
        Populations     &   8 cortical populations\\
        Topology        &   --\\
        Connectivity    &   Random connections with fixed number of synapses for 
                            each combination of pre- and postsynaptic population\\
        Neuron model    &   Leaky integrate-and-fire, fixed voltage threshold, fixed 
                            absolute refractory period\\
        Synapse model   &   exponential-shaped postsynaptic current\\
        Plasticity      &   --\\
        Input           &   Independent fixed-rate Poisson spike trains\\
        Measurements    &   Spike activity, membrane potentials \tnn

        \rowcolor{TableColor} Populations & \\
        Layers          &   L2/3, L4, L5, L6 \\
        Cortical network&   one excitatory (e) and one inhibitory (i) population per layer\\
        Size            &   population specific size 
                            (see table \ref{tab:network_params}) \tnn

        \rowcolor{TableColor} Connectivity & \\
        Type            &   Random connectivity with independently chosen pre- and postsynaptic
                            neurons; fixed total number of connections between two populations
                            (see table \ref{tab:network_params}) \\
        Weights         &   Fixed; drawn from clipped Gaussian distributions 
                            ($w > 0$ for excitatory, $w~<~0$ for inhibitory)\\
        Delays          &   Fixed; drawn from clipped Gaussian distributions ($d~>~0$);
                            multiples of computation step size \tnn

        \rowcolor{TableColor}\multicolumn{2}{l}{ Neuron and synapse model} \\
        Name            &   iaf neuron\\
        Type            &   Leaky integrate-and-fire, exponential-shaped current inputs\\
        Subthreshold \newline dynamics of \newline neuron~$i$
                        &   {$\!\begin{aligned} 
                            \tau_\text{m} \,\frac{\text{d} V_i(t)}{\text{d} t} 
                                    &= -(V_i(t) - E_\text{L}) + \frac{\tau_\text{m}}{C_\text{m}} I_i(t)
                                        &\text{if}\quad& t > t^* + \tau_\text{rp} \\ 
                                V_i(t)        &= V_\text{r}  &\text{else}& \\[0.2cm]
                                I_i(t) &= C_\text{m} \sum_j J_{ij} \sum_k \delta (t - t_j^k - d_{ij})  
                            \end{aligned}$}  \\
                            &THIS IS NOT TRUE -- STATE THE ACTUAL ODE! \\
        Spiking         &   If $\,\,V_i(t_-) < \theta \quad \land \quad V_i(t_+) \ge \theta$: \\
                        &   \quad 1. set $t^* = t$    \\
                        &   \quad 2. emit spike with time stamp $t^*$ \tnn

        \rowcolor{TableColor} Input & \\
        Type            &   Independent Poisson spikes to iaf neurons
                            (see table \ref{tab:network_params})
    \end{tabular}
\end{table}

% Network parameters
\begin{table}[htpb]
    \centering
    \caption{
        Network parameters
        }
    \label{tab:network_params}
    \begin{tabular}{p{3.5cm} *{8}{x{1.2cm}}}
        \rowcolor{TableColor}\multicolumn{9}{l}{Populations and inputs} \tn
        Name        
            & L23e & L23i & L4e & L4i & L5e & L5i & L6e & L6i  \tn
        Population size, $N$   
            & 20683 & 5834 & 21915 & 5479 & 4850 & 1065 & 14395 & 2948 \tn
        External inputs, $k_\text{ext}$ 
            & 1600 & 1500 & 2100 & 1900 & 2000 & 1900 & 2900 & 2100 \tn
        Background rate     
        & 8 Hz \tnn

        \rowcolor{TableColor}\multicolumn{9}{l}{Connection probabilities between pre- and postsynaptic populations} \tn
        \backslashbox{post}{pre}
            & L23e & L23i & L4e & L4i & L5e & L5i & L6e & L6i  \tn
        L23e
            & 0.101 & 0.169 & 0.044 & 0.082 & 0.032 & 0.000 & 0.008 & 0.000 \tn 
        L23i
            & 0.135 & 0.137 & 0.032 & 0.051 & 0.075 & 0.000 & 0.004 & 0.000 \tn 
        L4e
            & 0.008 & 0.006 & 0.050 & 0.135 & 0.007 & 0.000 & 0.045 & 0.000 \tn 
        L4i
            & 0.069 & 0.003 & 0.079 & 0.160 & 0.003 & 0.000 & 0.106 & 0.000 \tn 
        L5e
            & 0.100 & 0.062 & 0.051 & 0.006 & 0.083 & 0.373 & 0.020 & 0.000 \tn 
        L5i
            & 0.055 & 0.027 & 0.026 & 0.002 & 0.060 & 0.316 & 0.009 & 0.000 \tn 
        L6e
            & 0.016 & 0.007 & 0.021 & 0.017 & 0.057 & 0.020 & 0.040 & 0.225 \tn 
        L6i
            & 0.036 & 0.001 & 0.003 & 0.001 & 0.028 & 0.008 & 0.066 & 0.144 \tnn

        \rowcolor{TableColor}\multicolumn{9}{l}{Further connectivity} \tn
        $J \pm \delta J$    
            &  \multicolumn{3}{l}{$87.8 \pm 8.8 \,\text{pA}$}
            &  \multicolumn{5}{l}{Excitatory synaptic strengths} \tn
        $g$    
            &  \multicolumn{3}{l}{$-4$}
            &  \multicolumn{5}{l}{Relative inhibitory synapse strength} \tn
        $d_e \pm \delta d_e$    
            &  \multicolumn{3}{l}{$1.5 \pm 0.75 \,\text{ms}$}
            &  \multicolumn{5}{l}{Excitatory synaptic transmission delays} \tn
        $d_i \pm \delta d_i$    
            &  \multicolumn{3}{l}{$0.8 \pm 0.4 \,\text{ms}$}
            &  \multicolumn{5}{l}{Inhibitory synaptic transmission delays} \tnn

        \rowcolor{TableColor}\multicolumn{9}{l}{Neuron model} \tn
        $\tau_\text{m}$    
            &  \multicolumn{3}{l}{$10 \,\text{ms}$}
            &  \multicolumn{5}{l}{Membrane time constant} \tn
        $\tau_\text{ref}$    
            &  \multicolumn{3}{l}{$\hphantom{0}2 \,\text{ms}$}
            &  \multicolumn{5}{l}{Absolute refractory period} \tn
        $\tau_\text{syn}$    
        &  \multicolumn{3}{l}{$\hphantom{0}0.5 \,\text{ms}$}
            &  \multicolumn{5}{l}{Postsynaptic current time constant} \tn
        $C_\text{m}$    
            &  \multicolumn{3}{l}{$250 \,\text{pF}$}
            &  \multicolumn{5}{l}{Membrane capacity} \tn
        $E_\text{L}$    
            &  \multicolumn{3}{l}{$-65 \,\text{mV}$}
            &  \multicolumn{5}{l}{Leaky rest potential} \tn
        $V_\text{reset}$    
            &  \multicolumn{3}{l}{$-65 \,\text{mV}$}
            &  \multicolumn{5}{l}{Reset potential} \tn
        $\theta$    
            &  \multicolumn{3}{l}{$-50 \,\text{mV}$}
            &  \multicolumn{5}{l}{Fixed firing threshold} \tn
    \end{tabular}
\end{table}

\subsubsection{Synaptic input}
The spikes of neurons are modeled as delta functions. Accordingly, the 
input current of neuron $i$ can be described as the sum over arriving 
spike trains, 
\begin{equation}
    I_i(t) = C_\text{m} \sum_j J_{ij} \sum_k \delta (t - t_j^k - d_{ij}) \, ,
    \label{eq:input_current}
\end{equation}
where $t_j^k$ is the time the $k$-th spike by neuron $j$ was emitted and the 
delay between neuron $i$ and $j$ is set to $d_{ij}$. 


Explain synapse dynamics, the change from PSP to PSC, etc.


\subsection{Mean field model}
The derivation of a mean field theory for the layered network goes along the 
lines of the work by Brunel \cite{brunel2000}.
It starts of with a simplified model of 
$N$ neurons. Each neuron receives input from the network by $C$ synapses, $C_E$ and $C_I$ of 
which connect to excitatory and inhibitory neurons, respectively. 
Furthermore, each neurons receives $C_\text{ext} = C_E$ connections from 
external excitatory neurons.
The synapse numbers 
are determined by the relative size of the two populations through the factor
\begin{equation}
    \epsilon = \frac{C_E}{N_E} = \frac{C_I}{N_I} \,.
    \label{eq:epsilon}
\end{equation}
A central assumption is the sparsity of the network, expressed by $\epsilon \ll 1$.
Guided by anatomical estimates for the neocortex, the population sizes are set to
$N_e = 0.8N$ excitatory and $N_i = 0.2N$ inhibitory neurons. Taking \eqref{eq:epsilon}
this implies 
\begin{equation}
    C_I = \gamma C_E 	
 \label{eq:C_I}
\end{equation}
with $\gamma = 0.25$. The synaptic weights in this model are set to $J$ for 
excitatory presynaptic neurons and to $-g\, J$ for inhibitory ones. 
The delays are fixed uniformly to $d$ for all synapses. 

At the heart of the mean field model
is the transition from the deterministic description of membrane potential 
dynamics according to \eqref{eq:leaky_integrator} and \eqref{eq:input_current} to
a probabilistic formulation. Here, the input is modeled as a time-varying average part
$\mu(t)$ plus a fluctuating gaussian part with amplitude $\sigma(t)$:
\begin{equation}
    I_i(t) = C_\text{m} \tau_\text{m} \left[ \mu(t) + \sigma(t)	\sqrt{\tau_\text{m}} \eta_i(t) \right] \, .
    \label{eq:input_random}
\end{equation}
The random fluctuations are described by gaussian white noise $\eta_i(t)$ with 
$\langle  \eta_i(t)\rangle = 0$. The model explicitly excludes correlations, 
both in time and between different neurons, i.e.
\begin{equation}
    \langle \eta_i(t) \: \eta_j(t') \rangle = \delta_{ij} \: \delta(t - t')	\, . 
    \label{eq:no_correlations}
\end{equation}
The latter assumptions is by no means trivial and has to be tested for the spiking model
that is to be described with this mean field approach. 

The average input $\mu(t)$ and amplitude of fluctuations $\sigma(t)$ are linked to the 
average firing rate $\nu(t)$ by the equations
\begin{equation}
    \begin{split}
        \mu(t)          &= \mu_l(t) + \mu_\text{ext} \\
        \text{with} \qquad \mu_l(t)        &= C_E \, J (1 - \gamma g) \nu(t - d) \tau_\text{m} \\
        \mu_\text{ext}  &= C_E J \nu_\text{ext} \tau_\text{m}
        \label{eq:mu}
    \end{split}
\end{equation}
\begin{equation}
    \begin{split}
        \sigma^2(t)     &= {\sigma_l}^2(t) + {\sigma_\text{ext}}^2 \\
        \text{with} \qquad {\sigma_l}^2(t)       
                        &= C_E \, J^2 (1 + \gamma g^2) \nu(t - d) \tau_\text{m} \\
        {\sigma_\text{ext}}^2  &= C_E J^2 \nu_\text{ext} \tau_\text{m}
        \label{eq:sigma}
    \end{split}
\end{equation}

THIS IS EXTENDED TO:

\begin{align}
    \mu_a        &= 
        \tau \sum_{b \in \text{pop.}} C_{ab} \, J_{ab} \, \nu_b 
        + \tau C_{a, ext} \, J_{a, ext} \, \nu_{ext}
\intertext{and fluctuation}
    {\sigma_a}^2 &= 
        \tau \sum_{b \in \text{pop.}} C_{ab} \, {J_{ab}}^2  \, \nu_b
        +
        \tau C_{a, ext} \,{J_{a, ext}}^2 \,\nu_{ext}
\end{align}

\chapter{Results}
\label{sec:results}

\section{Spiking network model}
\paragraph{The spiking network model} will be analyzed in the following manner: 
At first, the results are directly compared to those obtained by the original 
model of Potjans and Diesmann. Both simulations are run for the same parameters, 
differences arise solely due to internal differences in assigning the random 
number generators. It is therefor not feasible to do a direct (spike per spike) 
comparison. Instead, a statistical description is chosen which furthermore sheds light
on the differences between individual realizations with different seeds for the 
random number generators.
The distribution of single neuron firing rate within the population
and the interspike interval distributions.
are then analyzed more thoroughly. 

A comparison between the PyNEST implementation and the original one written in SLI
is supplied in form of raster plots (\autoref{fig:raster_plot}), 
showing the activity of a subset of the network simulated for 400\,ms 
after a transient period of 200 ms. The subset shown consists of the first 
$2.5\,\%$ of the neurons of each population created during the initialization of the network. 
Both results have a very similar structure: 
The corresponding subsets show about the same number of events per time. This 
number varies strongly between different populations. Specifically, populations L2/3e and L6e 
fire at much lower frequencies compared to the remaining populations. 
Furthermore, both simulations show some signs of synchrony within populations, indicated by spikes 
of different neurons aligned in vertical lines. This is most apparent for the populations L2/3e, L2/3i, L4e and 
L5e.
\begin{figure}[tb]
    \centering
    \includegraphics{\figdir raster_plot}
    \caption[Raster plot: PyNEST and SLI]{
        Raster plot showing spontaneous activity of network for 
        (A) the PyNEST implementation and (B) the SLI implementation.
        The simulation and network parameters for both simulations are 
        the same. 
        For each layer, the excitatory population is the upper one shown 
        (total of 1924 neurons) for 400\,ms. 
    }
    \label{fig:raster_plot}
\end{figure}

A more thorough comparison is done on the bases of three statistical quantities: 
\autoref{fig:spontaneous_activity} shows the population means of single neurons firing rates, 
irregularity as measured by the CV of ISI as well as a measure for synchrony (the Fano factor of the 
PSTH) for both simulations (see Methods, \autoref{subsec:analysis} for details). 
All parameters were calculated from $20$ repetitions of a simulation for 
$60$ s.
Spikes were recorded from $1000$ neurons of each population, with recording 
starting after a transient period of $0.2$ s. 
For all three quantities, both implementations are in full agreement for the statistical fluctuations
observed. Fluctuations between different initializations differ for each quantities: For the firing rates, 
the largest variance is observed for L5e. For the irregularity, fluctuations are much smaller with the 
highest ones seen in populations L2/3e and L6e. Finally, the variation for synchrony is largest for 
L2/3e and L5e. For synchrony and firing rates, there is a tendency for the fluctuations to be larger for larger
values of the respective quantity -- note however the exception of L5i firing rate, which shows very little
fluctuation.
\begin{figure}[tb]
    \centering
    \includegraphics{\figdir spontaneous_activity}
    \caption[Spontaneous activity: PyNEST and SLI]{
        Measures of spontaneous activity for
        (\textbf{A - C}) the PyNEST implementation and (\textbf{D - E}) the SLI implementation, 
        using the same simulation and network parameters.
        Both implementations are run $20$ times independently (one marker per simulation),
        measuring 1000 spike trains of each population in a simulation for 60 s.
        In case of small 
        fluctuations, individual markers may not be identified due to overlap. 
        \quad (\textbf{A,~D}) Population mean of single neuron firing rates.
        \quad (\textbf{B,~E}) Irregularity of spike trains measured by the 
        population mean of CV of ISI.
        \quad (\textbf{C,~F}) Synchrony of populations quantified 
        by the Fano factor of the PSTH (bin width 3 ms).
    }
    \label{fig:spontaneous_activity}
\end{figure}

In order to get a deeper insight into the dynamics of the simulated network, the firing 
rates and the CV of ISI of single neurons are examined in \autoref{fig:single_neuron_activity}.
The observed fluctuations around the population mean are remarkably large and can 
be tracked back to the specific connection rule applied.
Note, however, that despite being large within a population, the 
fluctuations of the population means for different realizations remains small -- an observation 
important for the application of a mean field model. Both aspects will be reviewed in the 
Discussion, \autoref{sec:discussion}. 

\begin{figure}[tb]
    \centering
    \includegraphics{\figdir single_neuron_activity}
    \caption[Firing rates and CV of ISI of single neurons]{
        Single neuron activity measures:
        (\textbf{A}) 
        firing rates and
        (\textbf{B}) 
        CV of ISI 
        The data is take from one simulation of the PyNEST implementation, 
        corresponding to a single initialization in \autoref{fig:spontaneous_activity}.
        Statistical fluctuations 
        are indicated by the interquartile ranges (IQR) (boxes extend to 
        the first and third quartile). 
        The median is indicated by a black line, the population mean by a star and 
        whiskers extend to 1.5 IQR (outliers indicated by crosses). 
    }
    \label{fig:single_neuron_activity}
\end{figure}

\section{Mean field theory}
\paragraph{In the following section}, the results of the mean field theory are presented.
by showing the predicted firing rates and utilizing these rates 
to predict the previously introduced measure of irregularity
as well as the distribution of membrane potentials. 
These results are directly compared to 
the corresponding quantities recorded from the spiking network 
model simulation and analyzed in the previous section. 
The average input $\mu$ to each neuron is examined by separating 
the recurrent an external contribution. Finally, the possibility of  
applying the mean field model as a predictive tool is illustrated 
by calculating firing rates for varying relative inhibitory synapse 
strength $g$ and external frequency $\nu_\text{ext}$. 

The firing rates obtained by solving 
equation~\eqref{eq:self_consistency_a} are displayed in a bar plot in 
\autoref{fig:compare_sim_mf_fixed_total_number}. Rates measured in 
simulation and previously shown in \autoref{fig:spontaneous_activity}
are shown for comparison. As visible, the results of the mean field model 
match those of the simulation to a high degree:

The predicted rates cover those observed in simulation for the populations 
of all layers by L5. For the latter one, the rate of the excitatory population
is underestimated, the inhibitory one predicted slightly too high. 
Qualitatively, the sequence of populations ordered by increasing firing rates is reproduced.
The comparison can further be quantified: The difference 
$    \Delta \nu_a := \nu_{\text{mf}, a} - \nu_{\text{sim}, a} $
between the mean of simulated rates $\nu_{\text{sim}, a}$ and predicted rates 
$\nu_{\text{mf}, a}$ for each population $a$ is shown in \autoref{tab:diff_fixed_total_number}. 
The mean and standard 
deviation of the absolute values of $\Delta \nu_a$ 
are $(0.22 \pm  0.16)$ Hz. The relative difference
is largest for L2/3e with $-11\,\%$, while for all other populations 
the relative difference is smaller or equal $7\,\%$. 
The irregularity measured by the mean CV of ISI for each population is the result of 
plugging in the predicted rates into equation~\eqref{eq:CV_ISI_mf}. 
As for the rates, a comparison with simulation data is shown
in \autoref{fig:compare_sim_mf_fixed_total_number}, 
while numerical results are subsumed in 
\autoref{tab:diff_fixed_total_number}, using definitions analogous to those for the rates.
The theoretical results agree well: The relative difference between measured and
predicted values is lower than $10\,\%$ for all populations and again the order 
by sorting the populations according to increasing CV of ISI is reproduced. 
There is, however, a systematic overestimation of irregularity for all populations. 
The largest deviation is found for the populations L2/3e and L6e. For both populations, 
the mean field theory predicts a CV of ISI of almost one, corresponding to the case of 
a Poisson process. 

% Table of results of comparison
\begin{table}[htb]
    \centering
    \caption[Differences between prediction and simulation]{
        Difference between predicted and simulated population means for 
        firing rates and CV of ISI; absolute and relative to simulated quantities.}
    \label{tab:diff_fixed_total_number}
    \small
    \begin{tabular}{p{2.4cm} *{8}{x{0.87cm}}} \toprule
    \rowcolor{TableColor}
    \spacedlowsmallcaps{Population}
        & \mc2c{L2/3} & \mc2c{L4} & \mc2c{L5} & \mc2c{L6}  \tn
        \rowcolor{TableColor}
        & \mc1c{e} & \mc1c{i} & \mc1c{e} & \mc1c{i} & \mc1c{e} & \mc1c{i} & \mc1c{e} & \mc1c{i} \tn %\hline
        %Population $a$       
        %& L2/3e & L2/3i & L4e & L4i & L5e & L5i & L6e & L6i  \tn[0.2cm]
        \rule{0.0cm}{3ex} 
        $ \nu_{\text{mf}, a}$ / Hz
            &   0.82 &   3.02 &   4.64 &   6.12 &   7.14 &   8.92 &   1.04 &   8.09 \tn[0.2cm]
        $ \nu_{\text{sim}, a}$ / Hz
            &   0.92 &   3.00 &   4.40 &   5.84 &   7.70 &   8.65 &   1.10 &   7.84 \tn[0.2cm]
        $\Delta \nu_a$ / Hz
            &  -0.10 &   0.01 &   0.24 &   0.28 &  -0.56 &   0.27 &  -0.06 &   0.24 \tn[0.2cm]
        $\Delta \nu_a / \nu_{\text{sim}, a}$
            &  -0.11 & 4e-3&   0.05 &   0.05 &  -0.07 &   0.03 &  -0.05 &   0.03 \tn[0.4cm]
        $ \text{CV}_{\text{mf}, a}$
            &   0.99 &   0.94 &   0.92 &   0.91 &   0.89 &   0.84 &   0.99 &   0.85 \tn[0.2cm]
        $ \text{CV}_{\text{sim}, a}$
            &   0.92 &   0.92 &   0.89 &   0.88 &   0.84 &   0.81 &   0.91 &   0.81 \tn[0.2cm]
        $\Delta \text{CV}_a$
            &   0.07 &   0.03 &   0.03 &   0.03 &   0.05 &   0.04 &   0.08 &   0.04 \tn[0.2cm]
        $\Delta \text{CV}_a / \text{CV}_{\text{sim}, a}$
            &   0.07 &   0.03 &   0.03 &   0.04 &   0.06 &   0.05 &   0.08 &   0.05 \tn[0.2cm]

        \bottomrule
    \end{tabular}
\end{table}

% Comparison mean field / simulation
\begin{figure}[tb]
    \centering
    \includegraphics{\figdir compare_sim_mf_fixed_total_number}
    \caption[Comparing mean field model to simulation]{
        Comparison between mean field theory and spiking network model. 
        Bars indicate 
        (\textbf{A}) the single neuron firing rates and
        (\textbf{B}) the irregularity (mean CV of ISI)
        predicted by the mean field 
        theory, crosses the respective measurements from 20 simulation (as previously shown in
        \autoref{fig:spontaneous_activity}). The connection for the simulations
        rule was set to "fixed total number".
    }
    \label{fig:compare_sim_mf_fixed_total_number}
\end{figure}

Applying equation~\eqref{eq:P_V_a} for the predicted rates yields the 
membrane potential distributions shown in 
\autoref{fig:membrane_potential}. 
The obtained distributions are compared with the normalized histograms of recorded 
membrane potentials of a subpopulation of $n_\text{rec} = 100$ neurons for 
each population. The contribution due to neurons in refractory period is removed
(see Methods, \autoref{subsec:analysis} for details). 
The predictions agree well with the measured data: The overall shape, width and 
height of the distributions are recovered. For all populations, the maxima are
shifted toward the resting potential $V_\text{r} = -65$ mV. 
The effect
of neurons coming out of refractory period is underestimated in some cases, 
visible for example in populations L2/3i and L4i where the mean distributions 
show a step while the kink in the theoretical curves is hardly detectable. 

% Membrane potentials
\begin{figure}[tb]
    \centering
    \includegraphics{\figdir membrane_potential}
    \caption[Distribution of membrane potentials]{
        Distribution of membrane potentials for each population. 
        Shown are both the results of simulation (histogram) and 
        the predictions of the mean field theory (continuous line). 
        The simulation results are histograms (bins width $\Delta V_\text{m} = 0.25$ mV) 
        of membrane potential recordings 
        of 100 neurons, recorded every 0.001\,s for a simulation time of 1.0\,s 
        and adjusted for neurons in refractory period (see text). 
        The binning in voltage is the same as applied in Fig.~%
        \autoref{fig:single_membrane_potential}. 
        The voltage for neurons in refractory period $V_\text{r} = -65$\,mV 
        is indicated by the dashed and dotted line. The threshold is at 
        $\theta = -50$\,mV. 
    }
    \label{fig:membrane_potential}
\end{figure}

\paragraph{To illustrate} the average input $\mu$ to a neuron, the summands of the local input, cf.~%
equation~\eqref{eq:mu_a_plus}, the total local and external input $\mu_\text{local}$ and 
$\mu_\text{ext}$ as well as their sum are shown in \autoref{fig:input}.
One observes that the recurrent input is concentrated
mostly within layers, either on the diagonal or neighboring it. 
A notable exception is the input of layer 4 to neurons of layer 2/3.
Summing each row yields the total local input $\mu_\text{local}$ per neuron 
for each population. It is inhibitory for each population. L6e is inhibited most strongly, 
which is in agreement with the low firing rate. However, this observation has only
    limited meaning, as both external rates as well as fluctuations are not taken 
into account. The external input $\mu_\text{ext}$ balances the recurrent one, 
yielding positive values for all populations. In agreement with the firing rates, 
population L2/3e and L6e receive very little mean input. Contrary to this, 
population L6i receives even slightly more than L5i ($7.9$\,mV as opposed to $7.8$\,mV), 
while the latter one fires with a higher frequency. This shows the importance
of including the fluctuations. 

\begin{figure}[tb]
    \centering
    \includegraphics{\figdir input}
    \caption[Mean input $\mu$]{
        Mean input $\mu$ to neurons and separated 
        constituents (see equation~\eqref{eq:mu_a_plus}).
        (\textbf{A}) The local input of each presynaptic population $a$ 
        (columns) to a neurons op population $b$ (rows, postsynaptic). 
        The first column in (\textbf{B})
        corresponds to the total recurrent input $\mu_\text{local}$
        and equals the sum of each row in (\text{A}). The second column is 
        the external input $\mu_\text{ext}$.
        (\textbf{C}) Total input $\mu = \mu_\text{local} + \mu_\text{ext}$.
        Note the different scales between (\textbf{B}) and (\textbf{C}). 
    }
    \label{fig:input}
\end{figure}

We can now vary specific parameters in the mean field theory
and predict the network activity for the resulting new 
model. This is interesting, as a model used for explaining 
experimental data should be expected to be rather robust against
changes in specific parameters. One parameter that is estimated 
very differently in different theoretical models 
is the inhibitory synaptic 
strength $g$ (\citeb{sadeh2014mean} for example uses $g = 8$). 
In \autoref{fig:simulate_change_g}, the firing rates are calculated for $g$ 
varying over the range from $3$ to $10$. A first observation is the general
decrease of firing rates with increasing $g$. The individual populations, 
on the other hand, do not decrease at the same rate, such that the characteristic order 
of firing rates is changed. This is especially the case for population L4e, 
which stays almost constant on the entire range (remaining at the highest rate 
for very strong inhibition dominance) and L2/3e, for which the rate even 
increases. The strongest decrease is observed for layer L5e.
Note that \citeb{potjans2014} apply this change in $g$ in simulation. 
The data shown (Figure 8 of \cite{potjans2014}) agrees with the predictions
of the model, although comparison is limited to excitatory populations. 
For $g$ lower than $3$, the applied algorithm does not find a solution 
any more. For very low inhibition, however, the mean field model's assumptions
are not necessarily met any more, since individual firing rates tend to rise. 

\begin{figure}[tb]
    \centering
    \includegraphics{\figdir simulate_change_g}
    \caption[Firing rates for different $g$]{
        Firing rates for different inhibitory synaptic strength $g$
        predicted by the mean field theory. The working point of 
        the spiking network simulation is at $g = 4$, other theoretical
        models use higher estimates up to $g = 8$. 
    }
    \label{fig:simulate_change_g}
\end{figure}


\FloatBarrier

\section{Discussion}
\label{sec:discussion}

% Spiking network
The reimplementation of Potjans and Diesmann's spiking network simulation  
confirmed the expectation: The original model's results where reproduced 
well within the statistical fluctuations. The fluctuations of the calculated population 
means between different instantiations of the model are 
relatively small, a characteristic that can be interpreted as an indication that a mean 
field approach is suited well for describing the corresponding network:
The parameters describing the main features of the network activity
depend strongly on average input and connection numbers,
and much less on the actual wiring.
However, this assertion has to be restricted to 
random networks as more detailed structure, for example induce by learning, 
can lead to stronger effects of correlation (see for example \cite{staude2010higher}).
The fluctuations along neurons within one population and a single instantiation
of the network were shown to be large both for the firing rate and the 
CV of ISI. This can be tracked back to the large deviations of input between 
different neurons: Due to randomly choosing the synapses and adding further 
variability by distributing the synapse number (as shown for model validation, 
Methods, section \ref{subsec:methods_simulation}), 
some neurons will receive more excitatory or inhibitory input than others 
and fire accordingly. 
The measurement of the synchrony indicates that there is a significant 
amount of correlation among spike trains of different neurons. This has to be taken 
into account when interpreting the results of the mean field model. 

% Mean field theory
Introducing the mean field theory for the simulated spiking network 
turned out to be remarkably successful. 
The formal extension 
from the original model of two populations has been a rather small step, while
the numerical implementation revealed a number of obstacles. The resulting
algorithm, nonetheless, is a convenient and computationally inexpensive tool for 
predictions. It has been shown to predict central quantities 
of the network activity to a high degree of accuracy.  

% MF: Rates
The predicted population means of single neuron firing rates differ by 
$\sim 0.3$ Hz when compared to simulation data. For the populations firing at 
higher rates, i.~e. those of layers 4 and 5 as well as L6i, this corresponds 
to $<6 \%$ of the respective rates. Yet, for the more quiescent population L2/3e, 
the relative disagreement is much more drastic ($\sim 30 \%$). 
For all populations but L4e, the predicted rates are lower than the measured ones.  
One reason for this underestimation could be the negligence of correlation: 
Larger correlations among excitatory input can lead to higher fluctuations 
and thus higher spiking rates~\cite{staude2010higher}. This 
assertion has to be taken with care, though, as correlation in inhibitory 
populations can cancel the effect \emph{(citation!! Sadra emphasized that)}. 

% MF: CV of ISI
The prognosis for the irregularity of spike trains, measured by the coefficient of
variation of interspike intervals (CV of ISI) turns out to be more exact.:
Excluding the populations L2/3e and L6e, the deviation between theory and simulation 
is $<2 \%$. The two remaining 
populations are predicted to have the highest irregularity (as observed in simulation), 
and the deviation to measured values is about $7 \%$. 
One explanation for the remaining differences can be an estimation bias of the CV of ISI 
using spike trains of finite lengths arising as a significant ratio of the neurons in the 
population is either not included at all (if the number of spikes is $ < 2$) or the part of the distribution 
covering higher CV of ISI is not represented well (see \cite{nawrot2010analysis} for details). 
This is especially critical for the populations with low rates (L2/3e, L6e).
What cannot be accounted for by this bias would indicate a lower irregularity than that 
of the respective stationary Gaussian process, pointing towards temporal correlations
introduced for example by the synapse model 
(see e.~g. \cite{brunel1999fast} for the case of $\alpha$ synapses in a simpler context). 

% MF: Membrane potentials
The third and last measure predicted by the mean field model, the distribution of 
membrane potentials, shows more visible disagreement with the obtained simulation 
data than the previous ones. Although the general shape of the distributions is reproduced, the predicted
ones are narrower than the measured histograms. The maximum is further shifted 
towards the threshold. 
The kink at the resting potential $V_\text{r}$ due to neurons exiting the 
refractory period is reproduced but less pronounced than the measured one, indicating that the diffusion
away from this point is slower than assumed. The largest deviations are observed for 
the populations L2/3e and L6e, where the measured membrane potentials have a considerably 
high density below $V_\text{r}$.
Again, the fact that predicted rates are lower than the measured ones 
despite the distributions' means being closer to the threshold
would again point towards the impact of correlations:
Even if the membrane potential spends most time close to the resting potential, 
a large number of excitatory spikes arriving in a short time would lead to a quick rise and
firing without affecting the distribution significantly.  
This interpretation is in agreement with the measured degree of synchrony. 
Ultimately, the wide distribution of single neuron firing rates within each population
can be expected to widen the distribution. 

%The third and last measure predicted by the mean field model, the distribution of 
%membrane potentials, agrees well with the obtained simulation data, too. 
%Both the shape and position of the distribution are reproduced despite being slightly 
%narrower than the measured histograms. The kink at $V_\text{r}$ due to neurons exiting the 
%refractory period is reproduced but smaller than the measured one, indicating that the diffusion
%from this point is slower than assumed. The largest deviations are again observed for 
%the populations L2/3e and L6e, where the measured distributions are shifted towards 
%the resting potentials. For L4i, this is counterintuitive as the predicted rate is 
%lower than the measured one and would again point towards the impact of correlations:
%Even if the membrane potential spends most time close to the resting potential, 
%a large number of excitatory spikes arriving in a short time would lead to a quick rise and
%firing without deforming the distribution significantly.  

% Summary of results, explanations and shortcomings
In summary, the hypothesis for the analytical ansatz is confirmed: 
The considered activity measures of the simulated spiking network model of the neocortex 
can be predicted by a mean field theory assuming uncorrelated Gaussian input.
Possible explanations for the remaining deviations are (i) correlations between neurons, 
yielding an input to single neurons different from the assumed white Gaussian noise;
(ii) temporal correlations induced by the synapse type different from the 
delta synapses of the mean field model; and finally (iii) fluctuations in the 
input of single neurons. 
%\emph{Comparison with simulations where the according parameters are adapted 
    %have show the individual effects}.

% Use of the theory, limits
The value of the analytical framework developed is twofold: On the one hand, 
it provides a useful tool for predicting the activity of a spiking network model
over a large range of parameters 
%\emph{as observed for external frequency $\nu_{ext}$ 
%and relative inhibitory synapse strength $g$}. 
On the other hand, it represents an
essential means for identifying relevant measures and understanding the emergent 
dynamics of the complex systems under consideration. This is a hard and one of the 
most important tasks in this field. 
The long lasting
debate over whether neural coding is rate based or exploits precise timing and correlations
may at some point be solved by
excluding one or the other option using a sensible framework
of biological data, spiking network simulation and analytical arguments. 
On a less ambitious scale, the presented models can at least indicate by how much 
correlations effect the observed rates. 

% Outlook
The presented combination of the spiking network model and the mean field approach
could serve as a framework to tackle further questions. 
The implementation in PyNEST could work as a convenient basis for extensions 
such as the inclusion of newly available experimental data.
Furthermore, different neuron populations, 
especially concerning different interneuron classes, may be included, 
making the simulation a viable means for testing hypothesis about their 
role. 
Another possible extension already introduced in the 
original model \cite{potjans2014} is unspecific input from a thalamic population. 
When applying this input for for short bursts (e.~g. 10 ms), 
this can by used in order to examine a possibility for propagation of information
along the different layers in time and thus assessing the 
path of signal processing within the neocortex.
(\emph{see ??? for an overview of 
the presumed way of information processing over the layers})
The mean field approach, however, would have to be extended to a non-equilibrium 
regime since it lacks temporal resolution at the present state. 
When focusing on neural computation, one might also include specific input.
An interesting context is orientation selectivity, using 
oriented input resulting in neuronal tuning curves.
To this end, the mean field approach can be extended to single neurons 
as shown for example by Sadeh~\cite{sadeh2015orientation} 
for the case of one excitatory and one inhibitory population.





\chapter{Endnotes}
\label{sec:endnotes}
I am indebted to Stefan Rotter and Benjamin Merkt for introducing 
me into the fascinating field of computational neuroscience, always
being open for questions and triggering most fruitful discussions. I further want
to thank Jens Timmer for his supervision. Finally, I am grateful for the support 
of both my family and friends, without which this work would not have been possible.



\section{Bibliography}
\printbibliography[heading=subbibintoc]
\clearpage

\clearpage 
%\section{Additional figures}
\begin{figure}[htpb]
    \centering
    \includegraphics[width=0.8\linewidth]{\figdir population_activity}
    \caption{Population activity of all eight populations.
        (A) Number of spikes per 1 ms over the time of 1 s;
        (B) Power spectral density (PSC) of population activity minus mean activity. 
        Lighter and heavily oscillating background is the complete PSD, thick darker 
        line represents filtered data (Savitzky-Golay filter of 5th degree over 301 data points).
        The peaks indicate global oscillations at $\approx 82$ Hz (origin not clear).
    }
    \label{fig:population_activity}
\end{figure}

\begin{figure}[htpb]
    \centering
    \includegraphics[width=0.8\linewidth]{\figdir transition_mf_vs_model}
    \caption{Transition from Brunels model of equal populations to the neocortical network model. 
        Single neuron firing rates obtained by simulation (dotted lines) and 
        numerical solution of the mean field model. Distance $d$ indicates the 
        normalized $L^2$-norm in the parameter space ($8 \times 8$ synapse numbers $+ j_{02}$, see text
        for further explanation). 
    }
    \label{fig:transition_mf_vs_model}
\end{figure}



\end{document}

